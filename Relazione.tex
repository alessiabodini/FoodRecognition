\documentclass[11pt, a4paper, titlepage]{report}
\usepackage[utf8]{inputenc}
\usepackage[english, italian]{babel}
\usepackage{frontespizio}
\usepackage{hyperref, cleveref}
\hypersetup{hidelinks}
\usepackage{graphicx, wrapfig, subcaption, setspace, booktabs}
\graphicspath{{References//}}
\usepackage[margin=1in]{geometry}
\usepackage{adjustbox}
\usepackage{tabls}
\usepackage{tabularx}
\usepackage{listings}
\usepackage{enumerate}
\usepackage{float}
\usepackage{array}

\begin{document}

\begin{frontespizio}
\Margini{2.5cm}{3cm}{2.5cm}{3cm}
\Universita{Verona}
\Logo[4cm]{logo}
\Dipartimento{Informatica}
%\Facolta{Informatica}
\Corso[Laurea]{Ingegneria e scienze informatiche}
\Annoaccademico{2019--2020}
\Titoletto{Progetto di Teorie e Tecniche del Riconoscimento}
\Titolo{Food Recognition}
\Sottotitolo{}
%\NCandidato{Laureanda}
\Candidato[VR451051]{Alessia Bodini} 
\end{frontespizio}
%\IfFileExists{\jobname-frn.pdf}{}{%
%\immediate\write18{pdflatex \jobname-frn}}

%\title{Food Recognition}
%\author{Alessia Bodini}
%\date{\today}

%\maketitle
\tableofcontents
\newpage

% Chapter 1: Introduzione
\chapter{Introduzione}
Il progetto realizzato (\cite{github-link}) si basa sull'applicazione di un software di riconoscimento di scritte a un dataset di bottiglie di vino, e in particolare sulle loro etichette, con lo scopo di affibbiare a ognuna di esse il nome del vino ed eventuali informazioni aggiuntive (quali produttore e luogo di provenienza). 

Per l'analisi ci si è basati su un dataset contente immagini di vini suddivise in tre categorie:
\begin{itemize}
	\item \emph{bottles}\label{bottles}: immagini presenti in \emph{raw} appositamente ritagliate in modo da lasciare le bottiglie in primo piano.
	\medskip
	\begin{figure}[H]
	\centering 
	\includegraphics[width=0.15\textwidth]{CorteGiaraAmarone1}
	\caption{Bottiglia di Amarone di Corte Giara ritagliata da \cref{fig:CorteGiaraAmarone11}.}
	\end{figure}
\end{itemize}

A partire dal dataset, lo scopo del programma è quello di riconoscere all'interno di ogni immagine le parole presenti e, in seguito, di associare a ciascuna foto in \emph{raw} e \emph{bottles} il nome della bottiglia raffigurata, a seconda delle corrispondenze trovate con le immagini in \emph{gt}. All'interno di \emph{raw} e \emph{bottles} le immagini sono disposte in modo tale che ognuna di esse sia presente nella sottodirectory con il nome della bottiglia in \emph{gt} a cui fa riferimento, in modo da poter verificare facilmente l'accuratezza dei risultati raggiunti. 

\medskip
Un tale programma potrebbe essere adattato anche all'ambito mobile, così da consentire a possibili acquirenti di avere informazioni, possibilmente nella loro lingua madre, riguardanti il vino che sono interessati a comprare. 

\pagebreak


\bibliography{Bibliografia} 
\bibliographystyle{ieeetr}

\end{document}
