\documentclass[11pt, a4paper, titlepage]{article}
\usepackage[utf8]{inputenc}
\usepackage[english, italian]{babel}
\usepackage[noadvisor]{frontespizio}
\usepackage{hyperref, cleveref}
\hypersetup{hidelinks}
\usepackage{graphicx, wrapfig, subcaption, setspace, booktabs}
\graphicspath{{References//}}
\usepackage[margin=1in]{geometry}
\usepackage{adjustbox}
\usepackage{tabls}
\usepackage{tabularx}
\usepackage{listings}
\usepackage{enumerate}
\usepackage{float}

\begin{document}

\begin{frontespizio}
\Margini{2.5cm}{3cm}{2.5cm}{3cm}
\Universita{Verona}
\Logo[4cm]{logo}
\Dipartimento{Informatica}
%\Facolta{Informatica}
\Corso[Laurea]{Ingegneria e scienze informatiche}
\Annoaccademico{2019--2020}
\Titoletto{Progetto di Teorie e Tecniche del Riconoscimento}
\Titolo{Food Recognition}
\Sottotitolo{}
%\NCandidato{Laureanda}
\Candidato[VR451051]{Alessia Bodini} 
\end{frontespizio}
\IfFileExists{\jobname-frn.pdf}{}{
\immediate\write18{pdflatex \jobname-frn}}
%\lstinputlisting{Relazione-frn.log}

%\title{Food Recognition}
%\author{Alessia Bodini}
%\date{\today}

%\maketitle
\tableofcontents
\newpage

% MOTIVAZIONI
\section{Motivazioni e fondamento logico}
Il seguente progetto si pone lo scopo di identificare una serie di cibi facendo uso di modelli visti durante il corso di studio (KNN, SVM e NN). Tale tipo di riconoscimento può risultare molto utile per quanto riguarda la classificazione di piatti in tutto il mondo, ad esempio per viaggiotori o stranieri che vogliono avere maggiori informazioni sul piatto o per coloro che sono interessati ad conoscere i valori nutrizionali del cibo proposto, il tutto con una sola foto. 

% STATO DELL'ARTE
\section{Stato dell'arte}
L'applicazione maggiormente conosciuta per quanto riguardo il riconoscimento di cibi è probabilmente \emph{Calorie Mama}. Tale applicazione è disponibile per Apple e Android e permette non solo di riconoscere i cibi ma anche di mostrarne i valori nutrizionali e di far gestire all'utente le calorie assunte giornalmente e relativi programmi di fitness. La funzione di \textit{istant food recognition} viene alimentata dalla \textit{Food AI API} definita sulle ultime innovazioni in campo di deep learning e classificazione di immagini, costantemente aggiornata con cibi provenienti da tutto il mondo. Ogni piatto viene poi legato a specifici valori nutrizionali che l'utente utilizza per controllare le proprie diete direttamente dall'app.

\medskip
Il mio progetto non si pone di superare i risultati già raggiunti dall'applicazione, ma di eseguire un'analisi sulle migliori tecniche di classificazione conosciute e decretare la più efficiente tra queste. 

% OBBIETTIVI
\section{Obbiettivi}


\end{document}
