\documentclass[11pt, a4paper, titlepage]{article}
\usepackage[utf8]{inputenc}
\usepackage[english, italian]{babel}
\usepackage[noadvisor]{frontespizio}
\usepackage{hyperref, cleveref}
\hypersetup{hidelinks}
\usepackage{graphicx, wrapfig, subcaption, setspace, booktabs}
\graphicspath{{References//}}
\usepackage[margin=1in]{geometry}
\usepackage{adjustbox}
\usepackage{tabls}
\usepackage{tabularx}
\usepackage{listings}
\usepackage{enumerate}
\usepackage{float}
\usepackage{cite}

\begin{document}

\begin{frontespizio}
\Margini{2.5cm}{3cm}{2.5cm}{3cm}
\Universita{Verona}
\Logo[4cm]{logo}
\Dipartimento{Informatica}
%\Facolta{Informatica}
\Corso[Laurea]{Ingegneria e scienze informatiche}
\Annoaccademico{2019--2020}
\Titoletto{Progetto di Teorie e Tecniche del Riconoscimento}
\Titolo{Food Recognition}
\Sottotitolo{}
%\NCandidato{Laureanda}
\Candidato[VR451051]{Alessia Bodini} 
\end{frontespizio}
\IfFileExists{\jobname-frn.pdf}{}{
\immediate\write18{pdflatex \jobname-frn}}
%\lstinputlisting{Relazione-frn.log}

%\title{Food Recognition}
%\author{Alessia Bodini}
%\date{\today}

%\maketitle
\tableofcontents
\newpage

% MOTIVAZIONI
\section{Motivazioni e fondamento logico}
Il seguente progetto si pone lo scopo di identificare una serie di cibi facendo uso di modelli visti durante il corso di studio (KNN, SVM e NN). Tale tipo di riconoscimento può risultare molto utile per quanto riguarda la classificazione di piatti in tutto il mondo, ad esempio per viaggiotori o stranieri che vogliono avere maggiori informazioni sul piatto o per coloro che sono interessati ad conoscere i valori nutrizionali del cibo proposto, il tutto con una sola foto. 

% STATO DELL'ARTE
\section{Stato dell'arte}
L'applicazione maggiormente conosciuta per quanto riguardo il riconoscimento di cibi è al momento \emph{Calorie Mama} \cite{calorie-mama}. Tale applicazione è disponibile per Apple e Android e permette non solo di riconoscere i cibi ma anche di mostrarne i valori nutrizionali e di far gestire all'utente le calorie assunte giornalmente e relativi programmi di fitness. La funzione di \textit{istant food recognition} viene alimentata da \textit{Food AI API} \cite{foodai} basata sulle ultime innovazioni in campo di deep learning e in grado di riconoscere ad oggi 756 cibi diversi (gran parte cibi tipici di Singapore). Ogni piatto viene poi legato a specifici valori nutrizionali che l'utente utilizza per controllare le proprie diete direttamente dall'app.

% OBBIETTIVI
\section{Obbiettivi}
Il mio progetto non si pone di superare i risultati già raggiunti dall'applicazione nè da \textit{Food AI API}, ma di eseguire un'analisi sulle migliori tecniche di classificazione conosciute e decretare la più efficiente tra queste. In particolare il mio lavoro si è concentrato sull'analisi di tre principali metodi per la classificazione: KNN (\textit{K-Nearest Neighbors}), SVM (\textit{Support Vector Machine}) e reti neurali. 

\pagebreak

% METODOLOGIA 
\section{Metodologia}
Il lavoro si è suddiviso nella ricerca di un dataset e relativa estrazione dei dati e delle features poi utilizzate e nell'implementazione di alcuni dei modelli di riconoscimento visti durante il corso. Si spiegano di seguito nei dettagli tali processi. 

\subsection{Ricerca del dataset}
Il dataset scelto denominato \emph{Food-101} \cite{food-101} è disponibile sul sito \url{kaggle.com} e presenta un totale di 10100 fotografie di piatti e cibi diversi. In particolare, il dataset è suddiviso in 101 categorie di cibi, ognuno già etichettato, e presenta alcuni file HDF5 dai quali è possibile estrarre direttamente training e testing set con una risoluzione minore rispetto all'originale, in modo da velocizzare le operazioni. I dati utilizzati nelle vari modelli sono stati presi tutti dagli stessi due file, così da mantenere una certa coerenza con i risultati raggiunti:
\begin{itemize}
    \item \textit{food\_c101\_n10099\_r64x64x3.h5} per il training set, con 10099 immagini (con almeno una per categoria) con risoluzione 64x64x3 (RGB, uint8);
    \item \textit{food\_test\_c101\_n1000\_r64x64x3.h5} per il testing set, con 1000 immagini della stessa risoluzione indicata per il training set. 
\end{itemize}
 

\pagebreak
\bibliography{Bibliografia}{}
\bibliographystyle{plain}

\end{document}
